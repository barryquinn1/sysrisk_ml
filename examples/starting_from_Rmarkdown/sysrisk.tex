% Options for packages loaded elsewhere
\PassOptionsToPackage{unicode}{hyperref}
\PassOptionsToPackage{hyphens}{url}
%
\documentclass[
  10pt,
]{article}
\usepackage[]{mathpazo}
\usepackage{amssymb,amsmath}
\usepackage{ifxetex,ifluatex}
\ifnum 0\ifxetex 1\fi\ifluatex 1\fi=0 % if pdftex
  \usepackage[T1]{fontenc}
  \usepackage[utf8]{inputenc}
  \usepackage{textcomp} % provide euro and other symbols
\else % if luatex or xetex
  \usepackage{unicode-math}
  \defaultfontfeatures{Scale=MatchLowercase}
  \defaultfontfeatures[\rmfamily]{Ligatures=TeX,Scale=1}
\fi
% Use upquote if available, for straight quotes in verbatim environments
\IfFileExists{upquote.sty}{\usepackage{upquote}}{}
\IfFileExists{microtype.sty}{% use microtype if available
  \usepackage[]{microtype}
  \UseMicrotypeSet[protrusion]{basicmath} % disable protrusion for tt fonts
}{}
\makeatletter
\@ifundefined{KOMAClassName}{% if non-KOMA class
  \IfFileExists{parskip.sty}{%
    \usepackage{parskip}
  }{% else
    \setlength{\parindent}{0pt}
    \setlength{\parskip}{6pt plus 2pt minus 1pt}}
}{% if KOMA class
  \KOMAoptions{parskip=half}}
\makeatother
\usepackage{xcolor}
\IfFileExists{xurl.sty}{\usepackage{xurl}}{} % add URL line breaks if available
\IfFileExists{bookmark.sty}{\usepackage{bookmark}}{\usepackage{hyperref}}
\hypersetup{
  pdftitle={Herding to comply: Systemic risk consequences of capital policy actions in Europe},
  pdfauthor={Barry Quinn},
  pdfkeywords={Bayesian inference, Systemic risk, Capital policy, Unintended consequences},
  hidelinks,
  pdfcreator={LaTeX via pandoc}}
\urlstyle{same} % disable monospaced font for URLs
\usepackage[margin=1in]{geometry}
\usepackage{longtable,booktabs}
% Correct order of tables after \paragraph or \subparagraph
\usepackage{etoolbox}
\makeatletter
\patchcmd\longtable{\par}{\if@noskipsec\mbox{}\fi\par}{}{}
\makeatother
% Allow footnotes in longtable head/foot
\IfFileExists{footnotehyper.sty}{\usepackage{footnotehyper}}{\usepackage{footnote}}
\makesavenoteenv{longtable}
\usepackage{graphicx,grffile}
\makeatletter
\def\maxwidth{\ifdim\Gin@nat@width>\linewidth\linewidth\else\Gin@nat@width\fi}
\def\maxheight{\ifdim\Gin@nat@height>\textheight\textheight\else\Gin@nat@height\fi}
\makeatother
% Scale images if necessary, so that they will not overflow the page
% margins by default, and it is still possible to overwrite the defaults
% using explicit options in \includegraphics[width, height, ...]{}
\setkeys{Gin}{width=\maxwidth,height=\maxheight,keepaspectratio}
% Set default figure placement to htbp
\makeatletter
\def\fps@figure{htbp}
\makeatother
\setlength{\emergencystretch}{3em} % prevent overfull lines
\providecommand{\tightlist}{%
  \setlength{\itemsep}{0pt}\setlength{\parskip}{0pt}}
\setcounter{secnumdepth}{-\maxdimen} % remove section numbering
\usepackage{booktabs}
\usepackage{longtable}
\usepackage{array}
\usepackage{multirow}
\usepackage{wrapfig}
\usepackage{float}
\usepackage{colortbl}
\usepackage{pdflscape}
\usepackage{tabu}
\usepackage{threeparttable}
\usepackage{threeparttablex}
\usepackage[normalem]{ulem}
\usepackage{makecell}
\usepackage{xcolor}

\title{\emph{Herding to comply:} Systemic risk consequences of capital policy
actions in Europe}
\author{Barry Quinn}
\date{May 07, 2021}

\begin{document}
\maketitle
\begin{abstract}
In this paper, we contribute to the ongoing policy debate resilience of
financial institutions by assessing whether capital policy actions in
Europe contribute to systemic risk. Using a flexible Bayesian framework
we estimate systemic risk and assess the multilevel effects of capital
policy actions. This produces a hierarchy of systemic risk implications
of capital policy actions for each of 21 European countries. We focus
specifically on three categories of policy supervisor-defined action;
tightening, loosening, and ambiguous. Our results reveal that the
accumulation of tightening policy actions have the unintended
consequence of increasing system risk by between 1 and 10 quarterly
percentage points at the bank-level. When evaluating the intra-national
posterior probability of the `random effects', banks in Greece, Ireland
and the United Kingdom seem to be driving this result. Our results
suggest capital adequacy regulation can have the unintended consequence
of increasing systemic risk. This involuntary association may result
from several systemically unimportant institutions becoming `systemic as
a herd' when investing in the same asset classes to comply with capital
rules.
\end{abstract}

\hypertarget{introduction}{%
\section{Introduction}\label{introduction}}

Not since the Great Depression-era reforms has there been such sweeping
re-regulation of financial institutions and markets. Following the
global financial crisis of 2007-09, national regulators and
international bodies have promoted a series of reforms to foster
economic stability. Ten years on from the problem, regulators,
policymakers and industry participants are trying to assess the outcomes
of such reforms and the interplay of regulation, supervision and
financial stability. One of the key concerns is that individually
prudent behaviour of banks could amplify systemic risk. In particular,
regulators worry that reforms that aimed to restrict total balance sheet
risks, such as higher capital requirements and the introduction of
leverage ratio requirements under Basle III, could have resulted in
banks becoming increasingly similar as prudential rules have encouraged
expansion in certain areas and individual assets. As a result, while
banks might be safer individually, they might also have become
``systemic as a herd'', that is, susceptible to the same shocks, which
could, in turn, increase systemic risk. Despite its importance, work on
this topic is at a relatively early stage.

In this paper, we contribute to this debate by evaluating how European
banks' systemic risk contributions are affected by the build-up of
capital policy actions. We source the data on policy actions from the
\emph{Ma}cro\emph{P}rudential \emph{P}olicy \emph{E}valuation
\emph{D}atabase (MaPPED)\footnote{\url{https://www.ecb.europa.eu/pub/research/working-papers/html/mapped.en.html}}
introduced by Budnik and Kleibl (2018). MaPPED offers a detailed
overview of the ``\emph{life-cycle}'' of policy instruments that are
either genuinely macroprudential or essentially microprudential but
likely to impact the whole banking system significantly. It tracks
events of the evolution of eleven categories and 53 subcategories of
instruments. MaPPED uses a carefully designed questionnaire, in
cooperation with experts from national central banks and supervisory
authorities of all EU member states. Nevertheless, survey information
only reflects whether laws are on the books, not to what extent they
affect the network-wide risk in practice.

Building on the \(\Delta CoVaR\) methods (Adrian and Brunnermeier 2016;
Brunnermeier, Rother, and Schnabel 2020) , which allows us to study
banks as systemic \emph{risk inducers}, we employ a flexible two-stage
Bayesian empirical design. In stage one, we to capture the complete
posterior distribution of extreme event risk by extended the methods of
Bernardi, Gayraud, and Petrella (2013). Specifically, to improve
accuracy, we use a Bayesian adaptive LASSO technique to regularise the
set of time-varying covariates that adjust for macro-level dynamics in
tail risk. Unlikely a standard LASSO, this model adapts each predictor
separately in terms of the optimal regularisation to improve accuracy
and stabilise predictions(Li, Xi, and Lin 2010; Gelman, Hill, and
Vehtari 2020). In stage two, we use a Bayesian hierarchical regression
model to disentangle policy action effects. This multilevel framework
exploits the natural nested structure of the data (repeated observations
of banks within countries), allowing different intercepts and slopes to
varying across countries and over time. When the number of banks per
country is small, only including group indicators in a least-squares
regression gives unacceptably noisy estimates. Our bayesian multilevel
regression, via a partial pooling approach, uses the variation in the
data to estimate prior distribution on the deviation of the intercepts
and slopes (Gelman, Hill, and Vehtari 2020). The procedure is especially
appropriate when the critical variable of interest is at the country
level, and where some groupings (countries) have only a few publicly
listed banks.

Our baseline results point to a positive link between the build-up of
tightening policy actions and systemic risk contribution one quarter on.
Our hierarchical model disentangles this effect and finds that banks in
Greece, Ireland and the UK seem to be driving this risk inducing
behaviour. Overall, the influence of policy actions on individual system
risk is weak, with banks in many countries showing little impact from
policy actions one quarter. Interestingly, our models allow us to
compare effect sizes and loosening policy actions have the strongish
relative impact\footnote{Typically they have a 13 basis point reduction
  in average quarter systemic risk for a one standard deviation move in
  the predictor}. However, the estimate is quite noisy, with a 95\%
credibility interval of {[}-0.139,0.141{]}.

This finding supports the fallacy of composition hypothesis (Embrechts
et al. 2001): capital regulation - based on a bank's own risk - can
unintentionally exacerbate systemic risk (Acharya 2009; Danielsson,
Shin, and Zigrand 2012; Embrechts et al. 2001; Gehrig and Iannino 2017).
As banks comply with minimum capital standards, they invest (herd) in
similar securities that are more likely to be associated with common
factors, especially in crisis time (Danielsson, Shin, and Zigrand 2012).

There is a body of theoretical research which has investigated why banks
herd. One hypothesis posits that banks seek safety in similarity, that
is, as banks become concerned about regulatory bailouts, they have
incentives to copy each other's behaviour as they seek to exploit the
``too many to fail'' guarantee (Acharya and Yorulmazer 2007; Vives
2014). Another strand of the literature argues that banks herd through
investment choices. As individual banks diversify, their asset
portfolios may increase, increasing the linkages between financial
institutions and ultimately increasing systemic risk. Allen, Babus, and
Carletti (2012) present a model in which asset commonality and
short-term bank debt interact to generate systemic risk.

Herding behaviour has also been studied on the liabilities side (Farhi
and Tirole 2012; Horváth and Wagner 2017; Segura and Suarez 2011; Stein
2012). Vives (2014) argues that well-intentioned regulation and
supervision can lead to unexpected risk enhancing consequences through
strategic complementarity. The empirical literature has also
investigated banks herding behaviour through the diversification channel
(Hirakata, Kido, and Thum 2017; Nijskens and Wagner 2011).

Our study also contributes to the literature on bank risk-taking and the
quality of regulation and supervision. Studies using the World Bank
surveys on bank regulations and oversight find laws which empower remote
monitoring, promote information disclosure, and create incentives for
private sector corporate control advocate sustainable banks' performance
and encourage stability (Barth, Caprio, and Levine 2001, 2004, 2006,
2008, 2012). Further work focusing on bank risk-taking finds mixed
results. Agoraki, Delis, and Pasiouras (2011) find that the direct
effect of market power reduces risk-taking in banks in Central and
Eastern Europe. But, risk-taking is decreased only when banks with weak
(strong) market power are exposed to more stringent capital (activity
restrictions) regulation. They find official supervisory authority has
only a direct effect on bank risk. Klomp and Haan (2012) construct
multidimensional bank risk measures distilling 25 risk indicators into
two common factors. They see the relationship between bank risk,
regulation, and supervision depending on size, ownership structure, and
riskiness level. A few studies consider regulation, supervision and
system-wide risk. Demirgüç-Kunt and Detragiache (2011) find that BCP
compliance is not robustly associated with bank soundness, estimated by
a system-wide z-score. One shortcoming of this study is that the
authors' risk measure fails to capture systemic contribution at the bank
level, which in turn would not provide disaggregated information on
individual bank strategies (Delis and Staikouras 2011). Gehrig and
Iannino (2017) assess the Basel process of capital regulation's
financial stability using two risk measures that capture a bank's
contribution and exposure to systemic risk. Surprisingly, they find
evidence that the adoption of internal models of credit risk is
enhancing systemic risk. This effect is more pronounced in large
systemically critical European banks. Their results extended the finding
of unintended risk consequences of internal model-based regulation in
the German banking system (Behn, Haselmann, and Vig 2016). They provided
empirical evidence for Basel II endogenous systemic risk warnings of
(Embrechts et al. 2001).

The remainder of this paper is structured as follows. Section II
presents the data collection process. Section III offers a practical
design and estimation process. Section IV provides a discussion of the
essential findings, and Section V concludes.

\hypertarget{data}{%
\section{Data}\label{data}}

We start with daily equity and macroeconomic state variable data from
Refinitiv Datastream for the four ICB financial sector industries,
banks, investment banks and brokerage, insurance and real estate for 21
European countries. Quarterly balance sheet data used to create
bank-level predictors is gathered from Compustat Global. The quarterly
data is filtered to only include observations which have price to book
ratio and leverage values in the interval \([0,100]\). We further apply
a truncation to the maturity mismatch variable at the 1st and 99th
percentile. Finally, to ensure meaningful risk estimation, only those
institutions which have at least 2 years of equity data are included in
the sample. After data cleansing, we have a total of 724 financial
institutions in our (116 commercial banks, 238 investment banks, 284
real estate companies and 66 are insurance firms). The sample period for
the systemic risk estimation is 1995:I-2015:IV. To ensure reasonable
inferences from the relatively short sample period the main part of the
risk estimation is carried out using daily data.

We source the data on policy actions from the
\emph{Ma}cro\emph{P}rudential \emph{P}olicy \emph{E}valuation
\emph{D}atabase (MaPPED)\footnote{\url{https://www.ecb.europa.eu/pub/research/working-papers/html/mapped.en.html}}
introduced by Budnik and Kleibl (2018). MaPPED offers a detailed
overview of the ``\emph{life-cycle}'' of policy instruments which are
either genuinely macroprudential or are essentially microprudential but
likely to have a significant impact on the whole banking
system.\footnote{. It tracks events of the evolution of eleven
  categories and 53 subcategories of instruments} MaPPED was compiled
using a carefully designed questionnaire, in cooperation with experts
from national central banks and supervisory authorities of all EU member
states. We focus on the minimum capital requirements (hereafter MCR)
policy actions in this database. To investigate their relationship to
systemic risk, we create a quarterly sum of these policy actions
categorised by their intended impact. There are three categories of
impact:

\begin{enumerate}
\def\labelenumi{\arabic{enumi}.}
\tightlist
\item
  Policy loosening
\item
  Policy tightening
\item
  Other and ambiguous impact
\end{enumerate}

In line with Cerutti, Claessens, and Laeven (2017), we create a lagged
cumulative count of actions to capture the overall policy stance of the
regulators. This sum increases when they policy instrument is activated
or there is a change in scope and decreases when they are deactivated.
Unlike previous studies, we do not impose a prior assumption on the
impact of the three categories, as this is likely to bias any unintended
implications of each action\footnote{Previous studies have signed policy
  actions in terms of their intended consequence. In this way they are
  constructing a result rather than allow the data to provide evidence
  of any risk effect (Akinci and Olmstead-Rumsey 2017, @Cerutti2016,
  @Meuleman2019)}. Specifically, we capture the accumulated effect of
each policy action using a cumulative sum and hypothesis that:

\begin{enumerate}
\def\labelenumi{\arabic{enumi}.}
\tightlist
\item
  An increasing number of tightening actions will reduce systemic risk
\item
  An increasing number of policy loosening actions will have no effect
  on systemic risk
\item
  We have no a priori hypothesis on an increasing number of ambiguous
  actions.
\end{enumerate}

Table \ref{tab:tab1} shows the quarterly frequency of minimum capital
requirement policy actions by the national central bank's intended
impact. Policy tightening is the dominant intention over the sample
period, with 70\% of the total actions, while actions which loosen
policy are the least frequent.

In Table \ref{tab:tab2} we can further disentangle the three categories
in terms of the scope of the action. Of the 56 MCR actions categorised
as ambiguous the vast majority are labelled as a changes in the scope of
an existing tool. Again, policy tightening actions dominate with the
majority of actions being an activation of a policy tool, then a change
in scope, and finally a change in level of an existing policy. When
disentangling the specific descriptions of the 29 changes in level of
existing policy tightening actions, the textual sentiment is clearly an
increase in level tightness. This latter observation indicates the
extent of the aggressive attempts to curb excessive risk taking in
European banking. Our fundamental research question is to assess whether
this ramping up of actions had any unintended network consequences.

\begin{figure}
\centering
\includegraphics{../figures/paper-fig1-1.png}
\caption{Count of minimum capital requirements policy actions}
\end{figure}

Figure \ref{fig:fig1} describes the evolution of the MCR policy actions
over the sample period. The bottom panel indicates some spikes in
activity, notably in 2007 Q1, 2014 Q1, and 2015 Q1. These activity
spikes are unsurprising given intital reactions to the crisis and the
wave of regulatory reform which swept through Europe in this period.

\hypertarget{meth}{%
\section{Methodology}\label{meth}}

We use a Bayesian inference framework to estimate systemic risk and
investigate the impact of policy actions. In a dynamic setting, this
offers some benefits compared to classical panel estimators. In contrast
to traditional approaches, which treat group-levels parameters as
nuisance parameter to be removed, Bayesian methods allow group-level
estimates regularised to only include information that adds explanatory
power. Furthermore, they can be used to likely attenuation bias that is
known to be associated with policy action count data Akinci and
Olmstead-Rumsey (2017). Finally, they can include a noise component with
incorporates the panel structure of the data.

\hypertarget{systemic-risk-measurement}{%
\subsection{Systemic risk measurement}\label{systemic-risk-measurement}}

We use the \(CoVaR\) approach to systemic risk contribution estimation,
which extends the Value at Risk (VaR) concept to the system. CoVaR can
be thought of as the VaR of the whole system conditional on institution
i being in a certain state. Systemic risk is approximated using
\(\Delta CoVaR\); the difference between the \(CoVaR\) conditional on
the distress of an institution and the CoVaR conditional on the median
state of that institution. \(\Delta CoVaR\) is best thought of as a
reduced form statistical tool which captures tail codependency or the
part of systemic risk that co-moves with the distress of an institution.
\(\Delta CoVaR\) frames a bank as a \emph{risk inducer}, quantifying the
contribution of a financial institution to the system's level of
systemic risk. This is achieved by estimating the additional value at
risk of the entire financial system associated with this institution
experiencing distress (Brunnermeier, Rother, and Schnabel 2020).

Formally, Let \((Y_1,\dots,Y_d)\) be a d-dimensional random vector where
each \(Y_j\) is expressed through some covariates
\(X= (X_1,X_2,...,X_M)\). In the systemic risk context, \(Y_j\) denotes
the behaviour of either an institutions or the whole system. Without
loss of generality, thereafter, we fix \(\tau \in (0,1)\) and suppose
that we are interested in institutions j within system k. The
Value-at-Risk, \(VaR^{\bf{X},\tau}_j\) of institution j is the
\(\tau\)-th level conditional quantile of the random variable
\(Y_j|\bf{X}=x\):

\begin{equation}\label{eq:eq1}
\mathbb{P}(Y_j \leq VaR^{\bf{X},\tau}_j |\bf{X}=x)=\tau
\end{equation}

.The Conditional Value-at-Risk
\(\left( CoVaR_{system=k|j}^{\bf{X},\tau} \right)\)is the Value-at-Risk
of system k conditional on \(Y_j= VaR_j^{\bf{X}\tau}\) at the level
\(\tau\) which satisfies:

\begin{equation}\label{eq:eq2}
\mathbb{P}\left(Y_{system=k} \leq CoVaR_{system=k|j}^{\bf{X},\tau} |\bf{X}=x,Y_j=VaR^{\bf{X},\tau}_j \right)=\tau
\end{equation}

Using Bayesian inference, equation \ref{eq:eq2} implies that CoVaR
corresponds to the \(\tau\)-th percentile of the conditional
distribution (For a detailed exposition see Bernardi, Gayraud, and
Petrella 2013)

\hypertarget{bayesian-regularised-time-varying-delta-covar-estimation}{%
\subsection{\texorpdfstring{Bayesian regularised time-varying
\(\Delta CoVaR\)
estimation}{Bayesian regularised time-varying \textbackslash Delta CoVaR estimation}}\label{bayesian-regularised-time-varying-delta-covar-estimation}}

To capture all forms of risk including volatility feedback, adverse
asset price movements and funding liquidity risk, we estimate
\(\Delta CoVaR\) using daily return losses from 1995:I to 2018:IV for a
sample of European commercial banks, investment banks, insurance
companies and real estate companies. Return losses \(Y_{it}\) are
measured using market equity \(ME\) of the publicly traded institution,

\begin{equation}\label{eq:eq3}
Y_{i,t+1}=-log(ME_{i,t+1}/ME_{i,t})
\end{equation}

We extend the work of Bernardi, Gayraud, and Petrella (2013) by using an
Bayesian adaptive LASSO quantile regression to evaluate daily time
varying \(\Delta CoVaR\). Regularisation techniques, such as the LASSO,
have been shown to improve predictive accuracy of quantile regression by
only include information which adds to the predictive power of the
explanatory variable set (Li, Xi, and Lin 2010). Regularisation
techniques employed with Bayesian inference are increasing common in
financial econometric (See for example Tibshirani 2011; Mogliani and
Simoni 2020; Fan, Ke, and Wang 2020). The adaptive LASSO is especially
useful when estimated time-varying \(\Delta CoVaR\), given the set of
state variables used to adjust for macro-level risk dynamics (See Table
A.1 for full list). Finally, in the context of systemic risk, Bayesian
methods are highly flexible and are extreme useful in the context of the
analysis of interdependence effects of extreme market events. Using data
and prior information they provide the complete posterior distribution
of the parameters of interest. Since the quantities of interest in this
paper are risk measures, learning about the whole distribution becomes
more relevant due to the interpretation of \(CoVaR\) as financial losses
(Bernardi, Gayraud, and Petrella 2013)

Formally, the Bayesian Adaptive Lasso regression is a hierarchical model
which exploits a skewed Laplace distribution\footnote{which has the
  attractive property of being represented as a scale mixture of normal
  distributions (Tsionas 2003)} to estimated \(VaR^{\bf{X},\tau}_j\) and
\(\left( CoVaR_{system=k|j}^{\bf{X},\tau} \right)\) as follows:

\begin{equation}\label{eq:eq4}
Y_{jt}=\beta_{0,j}+\bf{M^{'}_{i,t-1}} \bf{\beta_j} +\theta_j z_j+\phi \xi_j \sqrt{\sigma^{-1}z_j} 
\end{equation}

\begin{equation}\label{eq:eq5}
Y_{kt}=\beta_{0k}+\bf{M^{'}_{i,t-1}} \bf{\beta_k} +\delta Y_{jt}+\theta z_j+\phi\xi_k\sqrt{\sigma^{-1}z_k}
\end{equation}

where \(M\) is a set of European financial and macroeconomic variables
detailed in Table A.1 in the appendix.

\hypertarget{var-and-covar-posterior-estimation}{%
\subsubsection{VaR and CoVaR posterior
estimation}\label{var-and-covar-posterior-estimation}}

In the Bayesian inference context, we summarise the simulations of the
parameter posterior distributions using maximisation of the posterior
density (Maximum a Posteriori or \(MaP\)), which has been shown to be
equivalent to minimisation problem in the frequentist context (Lin and
Chang 2012). We assume an asymmetric Laplace distribution for the error
terms and diffuse priors on the regressor parameters. From all the
\(MaP\) parameters involved in the marginal and conditional quantiles
the estimates of \(VaR^{\bf{X},\tau}_j\) and
\(\left( CoVaR_{system=k|j}^{\bf{X},\tau} \right)\) are derived as
follows:

\begin{equation}\label{eq:eq6}
(VaR^{X,\tau}_j)^{MaP} = X^{'}\beta_j^{MaP}
\end{equation} \begin{equation}\label{eq:eq7}
(CoVaR_{system=k|j}^{X^{'},\tau})^{MaP} = X^{'} \beta_{system=k}^{MaP}+\delta^{MaP}(VaR^{\bf{X},\tau}_j)^{MaP}
\end{equation}

\begin{figure}
\centering
\includegraphics{../figures/paper-fig2-1.png}
\caption{Posterior probability credible set MaP CoVaR estimates}
\end{figure}

Figure \ref{fig:fig2} visualised the Bayesian quantile adaptive \(L1\)
regularised estimate for a selection of banks. In each plot the
credibility set is represented by the shaded area and represents the
uncertainty in each MaP estimate. The x-axis is a daily time indicator
for the sample period. The plots summarise of these MaP estimates and
their 99\% credible sets for a number of the banks in the sample. The
shaded grey area characterises the uncertainty around the MaP estimate.
While the time patterns are similar in each plot with a spike at the
height of the European crisis, both KBC group and National Bank of
Greece stand out for their significantly higher systemic risk inducing
behaviour.

\hypertarget{bayesian-hierarchical-model}{%
\subsection{Bayesian Hierarchical
Model}\label{bayesian-hierarchical-model}}

This section will lay out the Bayesian hierarchical model which we use
to investigate systemic risk implications of capital actions in Europe.
Bayesian hierarchical or multilevel models exploit a data structure
which has a number of levels which are either nested or non-nested. The
key to identifying a variable as a level is that its units can be
regarded as a random sample from a wider population. In this instance,
we have a three level nested data structure of repeat observations
within banks within European countries. The banks are a random sample
from a wider population of banks likewise the countries are a random
sample from a wider population of countries. In general, multilevel
models are useful for exploring how relationships vary across
higher-level units\footnote{As a rule of thumb the highest level should
  have at least 20 units to be suitable for this type of analysis
  (\url{http://www.bristol.ac.uk/cmm/learning/multilevel-models/data-structures.html})}/
This is in stark contrast to traditional least squares regression models
where group effects are included as dummies, so called \emph{fixed
effects} models. When a predictor is defined at the group level,
e.g.~type of policy action applied by national regulator, a \emph{fixed
effects} model is unable to separate out effects due to observed and
unobserved group characteristics.\footnote{The effect of group-level
  predictors are confounded with the effects of the group dummies} To
date there are some notable exceptions in panel data econometrics to
this (see for example Wooldridge (2019)). There is no such issue with a
multilevel model and both types of variable can be estimated.

\hypertarget{multilevels-delta-covar-regressions}{%
\subsubsection{\texorpdfstring{multilevels \(\Delta CoVaR\)
Regressions}{multilevels \textbackslash Delta CoVaR Regressions}}\label{multilevels-delta-covar-regressions}}

The regression models use MaP estimates of \(\Delta CoVaR\) for the 116
commercial banks as the dependent variable. Our hierarchical regression
model aims to extract meaningful effects of country level policy actions
on bank level system risk levels. Our strategy allows for (bank-level)
coefficients on the policy action variables to vary by country and
quarter, in addition to being `fixed'. We can think of these then as our
three \emph{random effects}. Denoting \(y_{c,t[i]}\) as the
\(\widehat{CoVaR_{c,t[i]}}\) estimate of firm \(i\) in country in time
t, we set up two specifications. A baseline where we assume that policy
action effect across countries is constant. Then we relax this
assumption allowing banks in each country to experience their own effect
which is then regularised, via partially pooling, by the global mean
effect. Equation \ref{eq:eq8} describes our baseline model (regression
1), while \ref{eq:eq9} describes the more country-level \emph{random
effects} model (regression 2).

\begin{equation}\label{eq:eq8}
 y_{c,t[i]} = (\alpha + \alpha_{c,t}) + \beta^L loose_{c,t}+\beta^T tight_{c,t}+\beta^A ambig_{c,t}+Bank_Adjustments_{c,t[i]}+M_{c,t}+\epsilon_{c,t[i]}
\end{equation}

\begin{equation}\label{eq:eq9}
y_{c,t[i]} = (\alpha + \alpha_{c,t}) + (\beta^L+\beta^L_{c,t})loose_{c,t}+(\beta^T+\beta^T_{c,t})tight_{c,t} \\
  +(\beta^A+\beta^A_{c,t})ambig_{c,t}++Bank_Adjustments_{c,t[i]}+M_{c,t}+\epsilon
\end{equation}

\(\alpha, \beta^L, \beta^T\) and \(\beta_A\) are the `fixed' effect
population coefficients which represent the global `average' intercept
and globe slope coefficients for our policy actions variables for a
total pooled sample. \(\beta^L_{c,t},\beta^T_{c,t},\beta^A_{c,t}\) are
the `random' effects counterparts and represent the slop coefficients
for each of the 22 countries \(c\) and 83 quarters \(t\). The random
coefficient estimates for each policy impact variable, for each country
or year, deviates from the coefficient population average; such that
\(\beta^T_{c,t}\) describes how the systemic risk impact of tightening
policy actions in country \(c\) or year \(t\) deviates from the average
impact across taken all countries and quarters. Following Brunnermeier,
Rother, and Schnabel (2020) we use a concise set of bank-level and
macro-level adjustments\footnote{The common practice use of
  \emph{control} when describe the adjudicating variables in a
  regression implies overconfidence in the analyst's ability and is
  sloppy statistical language. The true meaning of a \emph{control}
  variables is when we can intervene and change the variable by a
  certain amount, which is clearly not possible in any observational
  studies. We therefore prefer the more modest and realistic
  \emph{adjusting} description}. The bank-level variables include: (i)
leverage, measured as the ratio of total assets to common equity; (ii)
maturity mismatch, measured as short-term borrowings to total assets;
(iii) size, measured as the log of market equity, (iv) loan growth
measures as the percentage change in gross loans, and (v) an asset price
boom indicator, which is the number of consecutive quarters of being in
the top decile of the price to book ratio across all firms. The macro
level variable set \(M\) is the same as that used in the time-varying
systemic risk estimation. Finally, we include a AR(1) error process to
adjust for the panel nature of our data, which is excluded from the
already crowded specification above.

\hypertarget{robustness-checks}{%
\subsection{Robustness checks}\label{robustness-checks}}

By its theoretical nature systemic risk,and its joint distribution with
policy actions, is a tail risk phenomenon, and capturing this feature in
the conditional distribution of our hierarchical regression may require
more nuance distributional assumes than normality. To this end we
consider three alternative probability distributions, skewed-normal,
t-distribution and log-normal. Widely accepted information criteria,
suggest that the t-distribution is the most appropriate among these
alternatives, which conforms with expert advice for modelling extreme
events in hierarchical models (Gelman et al. (2019)). The reported
results are those based on a t-distribution. Furthermore, attenuation
bias is a know issue when assess the impact of country level policy
actions on individual banks (Akinci and Olmstead-Rumsey 2017). The
flexibility of our Bayesian framework allows us to adjust for this
measurement error using a latent variable set up. The results using a
measurement error model are qualitatively similar to the reported
estimates and are available upon request.

\hypertarget{results}{%
\section{Results}\label{results}}

Table \ref{tab:tab3} and \ref{tab:tab4} present regression results for
our baseline and multilevel models. The baseline model assumes policy
actions effects are constant over time and across country. Predictors
are standardised. Bank-level covariate posterior probability are
obfuscated for brevity in the baseline model. In both instances the
Bayesian \(R^2\) indicates the model \emph{fit}\footnote{We follow the
  procedure outline in Gelman et al. (2019), which calculates the
  Bayesian Bayesian \(R^2\) as the variance of the predicted values
  divided by the variance of the predicted values plus the expected
  variance of the errors} is very good. For the baseline model Bayesian
\(R^2\) lies between \([0.781,0,793]\) and for the multilevel model lies
between \([0.885,0.891]\) for the 95\% credible interval. Both models
allows the effects to be correlated but we have not reported them in
this table as they are not significant. The results are stable to the
use of different priors and likelihoods.\footnote{Fixed effect
  coefficients are almost identical, while some non-critical variation
  in the random effects occurs. The Bayesian \(R^2\) for the normal
  likelihood is smaller than the student-t by around 9\%-13\%}.

Several initial findings are noteworthy. There is a statistically
credible positive link between the build up of tightening policy actions
and systemic risk one quarter on. Specifically our baseline model
estimates a global random effect of 0.058 with a 95\% posterior credible
interval of {[}0.011,0.105{]}. The estimates of equation \ref{eq:eq8}
provides a more complex set of outputs (this model has 473 parameters)
which are best summarised visually below\footnote{The direct frequentist
  counterpart to this model is a mixed effect model estimated by
  maximising a multivariate normal likelihood. Technically, these models
  to do not estimate the group-level parameters, but are treated as a
  random variable which is marginalised out to become part of the error
  term. Traditional ``fixed-effects'' panel data analytics perform a
  similar operation to remove group level parameters, although the
  group-level effects cannot be recovered, meaning the model coefficient
  represent a partial effect which ignores group-level covariance.}

\begin{figure}
\centering
\includegraphics{../figures/paper-fig3-1.png}
\caption{Posterior probability distributions for quarter random
intercepts}
\end{figure}

Figure \ref{fig:fig3} provides \emph{random} effects and their 99\%
\emph{plausibility} intervals for the quarterly intercept parameters for
the \(CoVaR^{99}_{it}\) regressions. The estimate show a clear risk
pattern which peaks in 2008:Q4; the epicenter of the recent financial
crisis in Europe.

\begin{figure}
\centering
\includegraphics{../figures/paper-fig4-1.png}
\caption{Posterior probability distributions for country random
intercepts}
\end{figure}

Figure \ref{fig:fig4} illustrates country level random intercepts and
show that banks in each country are experiencing meaningful different
systemic risk contributions hold all else equal.

\begin{figure}
\centering
\includegraphics{../figures/paper-fig5-1.png}
\caption{Posterior probability distributions for country level random
effects of tightening policy actions}
\end{figure}

\begin{figure}
\centering
\includegraphics{../figures/paper-fig6-1.png}
\caption{Posterior probability distributions of country level random
effects of loosening policy actions}
\end{figure}

\begin{figure}
\centering
\includegraphics{../figures/paper-fig7-1.png}
\caption{Posterior probability distributions for country level random
effects of ambiguous policy actions}
\end{figure}

Figures \ref{fig:fig5} \ref{fig:fig6} and \ref{fig:fig7} disentangle the
pooled effects of the baseline model into random effects at the country
level. Specifically, they capture the systemic risk implications one
quarter on of the for the build-up of policy tightening, loosening and
ambiguous actions respectively. The finding tightening policy actions
contributing to system risk can be disentangle to banks in Greece,
Ireland and the UK. The 99\% credible intervals in each of these
countries are statistically meaningful. More generally, the effect of
MCR policy actions on individual system risk are weak, with banks in
many countries showing little impact from policy actions one quarter on.
Interestingly, our models allows us to compare effect sizes and
loosening policy actions have the strongish relative impact (-0.13
average reduction on quarterly systemic risk for a 1 standard deviation
move in the predictor) although the estimate is quite noisy with a
credibility interval of {[}-0.139,0.141{]}.

\hypertarget{conclusion}{%
\section{Conclusion}\label{conclusion}}

The global financial crisis highlighted how losses at individual
financial institutions can spread across the financial system, giving
rise to systemic risk, and underscored the importance of regulation and
supervision to a well-functioning banking system. This paper assesses
the contribution of Europe capital adequacy policy actions a bank's
system-wide risk. We focus on the MaPPED database of European policy
actions. This offers a detailed overview of the ``life-cycle'' of policy
instruments which are either genuinely macroprudential or are
essentially microprudential but likely to have a significant impact on
the whole banking system. To hone in on our hypothesis question we
consider the subcategory of minimum capital requirement policy actions
and use a flexible bayesian hierarchical model to identify precise
effects of these policy action on systemic risk measured using Bayesian
\(\Delta CoVaR\). The MaP estimates from the posterior probabilities are
then used in the hierarchical regressions to investigate the capital
policy action impact.

Overall, our results point to a positive link between the build up of
tightening policy actions and systemic risk one quarter on. Specifically
our baseline model estimates a global random effect of 0.058 with a 95\%
posterior credible interval of {[}0.011,0.105{]}. Our hierarchical model
disentangle this result suggesting that banks in Greece, Ireland and the
UK seems to be driving this effect. Overall, the MCR policy actions have
a weak affect on systemic risk, with banks in many countries showing
little impact from policy actions one quarter on. Interestingly, our
models allows us to compare effect sizes and loosening policy actions
have the strongish relative impact (-0.13 reduction in average quarter
systemic risk for a 1 standard deviation move in the predictor) although
the estimate is quite noisy with a credibility interval of
{[}-0.139,0.141{]}.

Systemic risk can emanate from large institutions which are highly
interconnected but importantly several smaller institutions may be
systemic as a herd. We argue that during periods of increasing
regulatory pressure and compliance constraints, banks tend to choose
correlated risks and invest in correlated assets. This could increase
`herding'' as bank managers must benchmark themselves to regulatory
imposed industry standards. This type of market inefficiency could
increase, rather than decrease systemic risk.

\hypertarget{appendix}{%
\section{Appendix}\label{appendix}}

\hypertarget{table-a.1}{%
\subsubsection{Table A.1}\label{table-a.1}}

\begin{longtable}[]{@{}ccc@{}}
\toprule
\begin{minipage}[b]{0.36\columnwidth}\centering
Variable\strut
\end{minipage} & \begin{minipage}[b]{0.32\columnwidth}\centering
Description\strut
\end{minipage} & \begin{minipage}[b]{0.24\columnwidth}\centering
Frequency\strut
\end{minipage}\tabularnewline
\midrule
\endhead
\begin{minipage}[t]{0.36\columnwidth}\centering
Change in the three-month yield\strut
\end{minipage} & \begin{minipage}[t]{0.32\columnwidth}\centering
Measured as the change in the three-month Bund rate\strut
\end{minipage} & \begin{minipage}[t]{0.24\columnwidth}\centering
\strut
\end{minipage}\tabularnewline
\begin{minipage}[t]{0.36\columnwidth}\centering
Change in the slope of the yield\strut
\end{minipage} & \begin{minipage}[t]{0.32\columnwidth}\centering
Measured as the change in the spread between the long-term composite
bond and the three-month Treasury bill rate.\strut
\end{minipage} & \begin{minipage}[t]{0.24\columnwidth}\centering
\strut
\end{minipage}\tabularnewline
\begin{minipage}[t]{0.36\columnwidth}\centering
TED spread\strut
\end{minipage} & \begin{minipage}[t]{0.32\columnwidth}\centering
Measured as the difference between the three-month EUBOR rate and the
three-month secondary market bund rate.\strut
\end{minipage} & \begin{minipage}[t]{0.24\columnwidth}\centering
Refinitiv\strut
\end{minipage}\tabularnewline
\begin{minipage}[t]{0.36\columnwidth}\centering
Change in the credit spread\strut
\end{minipage} & \begin{minipage}[t]{0.32\columnwidth}\centering
Measured as the change in the spread between the 10-year BAA rated bonds
and the 10-year Treasury bonds.\strut
\end{minipage} & \begin{minipage}[t]{0.24\columnwidth}\centering
Refinitiv\strut
\end{minipage}\tabularnewline
\begin{minipage}[t]{0.36\columnwidth}\centering
Europe market returns\strut
\end{minipage} & \begin{minipage}[t]{0.32\columnwidth}\centering
Daily\strut
\end{minipage} & \begin{minipage}[t]{0.24\columnwidth}\centering
Refinitiv\strut
\end{minipage}\tabularnewline
\begin{minipage}[t]{0.36\columnwidth}\centering
Daily housing sector excess returns\strut
\end{minipage} & \begin{minipage}[t]{0.32\columnwidth}\centering
Daily\strut
\end{minipage} & \begin{minipage}[t]{0.24\columnwidth}\centering
Refinitiv\strut
\end{minipage}\tabularnewline
\begin{minipage}[t]{0.36\columnwidth}\centering
Equity volatility\strut
\end{minipage} & \begin{minipage}[t]{0.32\columnwidth}\centering
Which is computed as the 22-day rolling standard deviation of daily
equity market returns\strut
\end{minipage} & \begin{minipage}[t]{0.24\columnwidth}\centering
Daily\strut
\end{minipage}\tabularnewline
\bottomrule
\end{longtable}

\hypertarget{references}{%
\section{References}\label{references}}

\begin{figure}
\centering
\includegraphics{../figures/paper-sample-plot-1.png}
\caption{Tea and Biscuits are correlated but there may be
identifiability issues.}
\end{figure}

Lorem ipsum dolor sit amet, consectetur adipisicing elit, sed do eiusmod
tempor incididunt ut labore et dolore magna aliqua. Ut enimad minim
veniam, quis nostrud exercitation ullamco laboris nisi ut aliquip ex ea
commodo consequat. Duis aute irure dolor in reprehenderit in voluptate
velit esse cillum dolore eu fugiat nulla pariatur. Excepteur sint
occaecat cupidatat non proident, sunt in culpa qui officia deserunt
mollit anim id est laborum.

\begin{itemize}
\tightlist
\item
  c(-0.00313977556330446, 0.432900018326591, 0.0739771863305975,
  0.0439566933654575, -0.042442484217676, 9.8483299170719,
  0.966232315489912, 2.58281915454164e-16, -0.149945094212485,
  0.345669390935031, 0.143665543085876, 0.52013064571815)
\item
  c(0.00944813491364757, 0.398395949603185, 0.100384027953451,
  0.0716865364287312, 0.0442747157236682, 0.0359932602314417,
  0.131797899359256, 8.99827233425268, 2.78896735966587,
  0.895417140279786, 1.95019640158856e-14, 0.00636489893887212,
  -0.132829796061491, 0.310522894752243, 0.0289473698318732,
  0.151726065888786, 0.486269004454127, 0.171820686075028)
\item
  c(-0.0489261053527959, 0.318582161990091, 0.0623259535838793,
  0.224875162679086, 0.0601308582816274, 0.0386306642756723,
  0.0304082169392583, 0.0335877174831526, -0.813660518924358,
  8.2468724771766, 2.04964183557286, 6.69516059827174,
  0.417853014898847, 8.50941534783873e-13, 0.0431267845691368,
  1.45073087144776e-09, -0.168284915680207, 0.24190089945919,
  0.00196612001808005, 0.15820407041518, 0.0704327049746146,
  0.395263424520992, 0.122685787149679, 0.291546254942993)
\end{itemize}

\hypertarget{conclusion-1}{%
\section{Conclusion}\label{conclusion-1}}

Lorem ipsum dolor sit amet, consectetur adipisicing elit, sed do eiusmod
tempor incididunt ut labore et dolore magna aliqua. Ut enimad minim
veniam, quis nostrud exercitation ullamco laboris nisi ut aliquip ex ea
commodo consequat. Duis aute irure dolor in reprehenderit in voluptate
velit esse cillum dolore eu fugiat nulla pariatur. Excepteur sint
occaecat cupidatat non proident, sunt in culpa qui officia deserunt
mollit anim id est laborum.

\hypertarget{references-1}{%
\section{References}\label{references-1}}

\setlength{\parindent}{-0.2in}
\setlength{\leftskip}{0.2in}
\setlength{\parskip}{8pt}
\vspace*{-0.2in}

\noindent

\hypertarget{refs}{}
\leavevmode\hypertarget{ref-Acharya2009}{}%
Acharya, Viral V. 2009. ``A Theory of Systemic Risk and Design of
Prudential Bank Regulation.'' \emph{Journal of Financial Stability} 5
(3): 224--55.

\leavevmode\hypertarget{ref-Acharya2007}{}%
Acharya, Viral V, and Tanju Yorulmazer. 2007. ``Too Many to Fail---an
Analysis of Time-Inconsistency in Bank Closure Policies.'' \emph{Journal
of Financial Intermediation} 16 (1): 1--31.

\leavevmode\hypertarget{ref-Adrian2016}{}%
Adrian, Tobias, and Markus K Brunnermeier. 2016. ``CoVaR.'' \emph{Am.
Econ. Rev.} 106 (7): 1705--41.

\leavevmode\hypertarget{ref-Agoraki2011}{}%
Agoraki, Maria-Eleni K, Manthos D Delis, and Fotios Pasiouras. 2011.
``Regulations, Competition and Bank Risk-Taking in Transition
Countries.'' \emph{Journal of Financial Stability} 7 (1): 38--48.

\leavevmode\hypertarget{ref-Akinci2017}{}%
Akinci, Ozge, and Jane Olmstead-Rumsey. 2017. ``How Effective Are
Macroprudential Policies? An Empirical Investigation.'' \emph{Journal of
Financial Intermediation}, April.

\leavevmode\hypertarget{ref-Allen2012}{}%
Allen, Franklin, Ana Babus, and Elena Carletti. 2012. ``Asset
Commonality, Debt Maturity and Systemic Risk.'' \emph{J. Financ. Econ.}
104 (3): 519--34.

\leavevmode\hypertarget{ref-Barth2001}{}%
Barth, James, Gerard Caprio, and Ross Levine. 2001. ``Bank Regulation
and Supervision: A New Database.'' \emph{Brookings-Wharton Papers on
Financial Services}.

\leavevmode\hypertarget{ref-Barth2004}{}%
Barth, James R, Gerard Caprio Jr., and Ross Levine. 2004. ``Bank
Regulation and Supervision: What Works Best?'' \emph{Journal of
Financial Intermediation} 13 (2): 205--48.

\leavevmode\hypertarget{ref-Barth2006}{}%
Barth, James R, Gerard Caprio, and Ross Levine. 2006. \emph{Rethinking
Bank Regulation: Till Angels Govern}. Edited by Govern, Till, and
Angels. Second edi. New York: Cambridge University Press.

\leavevmode\hypertarget{ref-Barth2008}{}%
---------. 2008. ``Bank Regulations Are Changing: For Better or Worse?''
\emph{Comp. Econ. Stud.} 50 (4): 537--63.

\leavevmode\hypertarget{ref-Barth2012}{}%
---------. 2012. \emph{Guardians of Finance: Making Regulators Work for
Us}. MIT Press.

\leavevmode\hypertarget{ref-Behn2016}{}%
Behn, Markus, Rainer F H Haselmann, and Vikrant Vig. 2016. ``The Limits
of Model-Based Regulation,'' July.

\leavevmode\hypertarget{ref-Bernardi2013}{}%
Bernardi, Mauro, Ghislaine Gayraud, and Lea Petrella. 2013. ``Bayesian
Inference for CoVaR,'' June. \url{http://arxiv.org/abs/1306.2834}.

\leavevmode\hypertarget{ref-Brunnermeier2020}{}%
Brunnermeier, Markus, Simon Rother, and Isabel Schnabel. 2020. ``Asset
Price Bubbles and Systemic Risk.'' \emph{Rev. Financ. Stud.} 33 (9):
4272--4317.

\leavevmode\hypertarget{ref-Budnik2018}{}%
Budnik, Katarzyna Barbara, and Johannes Kleibl. 2018. ``Macroprudential
Regulation in the European Union in 1995-2014: Introducing a New Data
Set on Policy Actions of a Macroprudential Nature.'' \emph{European
Central Bank Working Paper Series}, no. 2123 (January).

\leavevmode\hypertarget{ref-Cerutti2017}{}%
Cerutti, Eugenio, Stijn Claessens, and Luc Laeven. 2017. ``The Use and
Effectiveness of Macroprudential Policies: New Evidence.'' \emph{Journal
of Financial Stability} 28 (February): 203--24.

\leavevmode\hypertarget{ref-Cerutti2016}{}%
Cerutti, Mr Eugenio M, Mr Ricardo Correa, Elisabetta Fiorentino, and
Esther Segalla. 2016. \emph{Changes in Prudential Policy Instruments ---
a New Cross-Country Database}. International Monetary Fund.

\leavevmode\hypertarget{ref-Danielsson2012}{}%
Danielsson, Jon, Hyun Song Shin, and Jean-Pierre Zigrand. 2012.
``Endogenous and Systemic Risk.'' In \emph{Quantifying Systemic Risk},
73--94. University of Chicago Press.

\leavevmode\hypertarget{ref-Delis2011}{}%
Delis, Manthos D, and Panagiotis K Staikouras. 2011. ``Supervisory
Effectiveness and Bank Risk.'' \emph{Rev Financ} 15 (3): 511--43.

\leavevmode\hypertarget{ref-Demirguc-Kunt2011}{}%
Demirgüç-Kunt, Asli, and Enrica Detragiache. 2011. ``Basel Core
Principles and Bank Soundness: Does Compliance Matter?'' \emph{Journal
of Financial Stability} 7 (4): 179--90.

\leavevmode\hypertarget{ref-Embrechts2001}{}%
Embrechts, Paul, Jon Danielsson, Charles A E Goodhart, Con Keating,
Felix Muennich, Olivier Renault, and Hyun Song Shin. 2001. ``An Academic
Response to Basel II.'' \emph{FMG Special Paper 130} 130.

\leavevmode\hypertarget{ref-Fan2020}{}%
Fan, Jianqing, Yuan Ke, and Kaizheng Wang. 2020. ``Factor-Adjusted
Regularized Model Selection.'' \emph{J. Econom.} 216 (1): 71--85.

\leavevmode\hypertarget{ref-Farhi2012}{}%
Farhi, Emmanuel, and Jean Tirole. 2012. ``Collective Moral Hazard,
Maturity Mismatch, and Systemic Bailouts.'' \emph{Am. Econ. Rev.} 102
(1): 60--93.

\leavevmode\hypertarget{ref-Gehrig2017}{}%
Gehrig, Thomas, and Maria Chiara Iannino. 2017. ``Did the Basel Process
of Capital Regulation Enhance the Resiliency of European Banks?''
\emph{CEPR Discussion Paper No. DP11920}.

\leavevmode\hypertarget{ref-Gelman2019}{}%
Gelman, Andrew, Ben Goodrich, Jonah Gabry, and Aki Vehtari. 2019.
``R-Squared for Bayesian Regression Models.'' \emph{Am. Stat.} 73 (3):
307--9.

\leavevmode\hypertarget{ref-Gelman2020}{}%
Gelman, Andrew, Jennifer Hill, and Aki Vehtari. 2020. \emph{Regression
and Other Stories}. Cambridge University Press.

\leavevmode\hypertarget{ref-Hirakata2017}{}%
Hirakata, Naohisa, Yosuke Kido, and Jie Liang Thum. 2017. ``Empirical
Evidence on `Systemic as a Herd': The Case of Japanese Regional Banks.''
\emph{Bank of Japan Working Paper Series}.

\leavevmode\hypertarget{ref-Horvath2017}{}%
Horváth, Bálint L, and Wolf Wagner. 2017. ``The Disturbing Interaction
Between Countercyclical Capital Requirements and Systemic Risk.''
\emph{Rev Financ} 21 (4): 1485--1511.

\leavevmode\hypertarget{ref-Klomp2012}{}%
Klomp, Jeroen, and Jakob de Haan. 2012. ``Banking Risk and Regulation:
Does One Size Fit All?'' \emph{Journal of Banking \& Finance} 36 (12):
3197--3212.

\leavevmode\hypertarget{ref-Li2010}{}%
Li, Qing, Ruibin Xi, and Nan Lin. 2010. ``Bayesian Regularized Quantile
Regression.'' \emph{Bayesian Anal.} 5 (3): 533--56.

\leavevmode\hypertarget{ref-Lin2012}{}%
Lin, Nan, and Chao Chang. 2012. ``Comment on Article by Lum and
Gelfand.'' \emph{Bayesian Analysis}.

\leavevmode\hypertarget{ref-Meuleman2019}{}%
Meuleman, Elien, and Rudi Vander Vennet. 2019. ``Macroprudential Policy
and Bank Systemic Risk.''

\leavevmode\hypertarget{ref-Mogliani2020}{}%
Mogliani, Matteo, and Anna Simoni. 2020. ``Bayesian MIDAS Penalized
Regressions: Estimation, Selection, and Prediction.'' \emph{J. Econom.},
August.

\leavevmode\hypertarget{ref-Nijskens2011}{}%
Nijskens, Rob, and Wolf Wagner. 2011. ``Credit Risk Transfer Activities
and Systemic Risk: How Banks Became Less Risky Individually but Posed
Greater Risks to the Financial System at the Same Time.'' \emph{Journal
of Banking \& Finance} 35 (6): 1391--8.

\leavevmode\hypertarget{ref-Segura2011}{}%
Segura, Anatoli, and Javier Suarez. 2011. ``Liquidity Shocks, Roll-over
Risk and Debt Maturity.'' \emph{CEPR Discussion Paper No. DP8324},
April.

\leavevmode\hypertarget{ref-Stein2012}{}%
Stein, Jeremy C. 2012. ``Monetary Policy as Financial Stability
Regulation.'' \emph{Q. J. Econ.} 127 (1): 57--95.

\leavevmode\hypertarget{ref-Tibshirani2011}{}%
Tibshirani, Robert. 2011. ``Regression Shrinkage and Selection via the
Lasso: A Retrospective.'' \emph{J. R. Stat. Soc. Series B Stat.
Methodol.} 73 (3): 273--82.

\leavevmode\hypertarget{ref-Tsionas2003}{}%
Tsionas, Efthymios G. 2003. ``Bayesian Quantile Inference.'' \emph{J.
Stat. Comput. Simul.} 73 (9): 659--74.

\leavevmode\hypertarget{ref-Vives2014}{}%
Vives, Xavier. 2014. ``Strategic Complementarity, Fragility, and
Regulation.'' \emph{Rev. Financ. Stud.} 27 (12): 3547--92.

\leavevmode\hypertarget{ref-Wooldridge2019}{}%
Wooldridge, Jeffrey M. 2019. ``Correlated Random Effects Models with
Unbalanced Panels.'' \emph{J. Econom.} 211 (1): 137--50.

\end{document}
